\documentclass[11pt]{article}
\usepackage{fullpage}
\usepackage{amsmath}
\usepackage{esint}
\usepackage{cancel}
\usepackage{graphicx}
\linespread{1.1}


\begin{document}

\title{ECE259 \\ Electricity and Magnetism}
\author{Michael Boyadjian}
\maketitle
\pagebreak

\tableofcontents

\pagebreak

\bigskip
\bigskip
\bigskip

\section{Static Electric Fields}

\subsection{Fundamental Postulates of Electrostatics in Free Space}
Electric field intensity is defined as the force per unit charge that a very small stationary test charge experiences when it is placed in a region where and electric field exists:
$$ \vec{E} = \lim_{q \to \infty} \frac{\vec{F}}{q} \quad \left[\frac{V}{m}\right] $$
This is proportional to and in the direction of the force $\vec{F}$. We can express the inverse relationship, which gives the force $\vec{F}$ on a charge $q$ in an electric field $\vec{E}$ \\ \\
The two fundamental postulates of electrostatics in free space specify the \textbf{divergence} and \textbf{curl} of $\vec{E}$
$$ \vec{ \nabla}  \cdot \vec{E} = \frac{\rho}{\epsilon_0} \quad \quad \quad \vec{ \nabla}  \times \vec{E} =0 $$
These two expression are point relations, and are referred to as the differential form of the postulates of electrostatics. We can integrate to get the expressions for a total field of an aggregate or a distribution of charges.
$$ \oint_{S} \vec{ E}  \cdot ds = \frac{Q}{\epsilon_0}\quad \quad \quad \oint_{C} \vec{ E}  \cdot dl = 0 $$


\subsection{Coulomb's Law}
We can express the electric field intensity on a point charge as the following:
$$ \vec{E} = \vec{a_R}E_R = \vec{a_R} \frac{q}{4 \pi \epsilon_0 R^2} \quad \left[\frac{V}{m}\right]$$
Here we assume the charge is located at the origin, but this is often not the case. If we want to account for the changes in $\vec{a_R}$ we can use the vector form of the equation above:
$$ \vec{E_p} =  \frac{q (\vec{R} - \vec{R'})}{4 \pi \epsilon_0 |\vec{R} - \vec{R'}|^3} \quad \left[\frac{V}{m}\right]$$
We are also able to determine the force $\vec{F_{12}}$ experienced by a charge $q_2$ when placed in the field of another charge $q_1$. This is in fact the mathematical form of Coulomb's Law:
$$ \vec{F_{12}} = q_2\vec{E_{12}} = \vec{a_R} \frac{q_1q_2}{4\pi\epsilon_0R^2} \quad [N]$$

\subsubsection{Electric Field Due to a System of Discrete Charges}
All of these equations presented so far involve only a single charge, but what if we have a field created from many point charges. Since electric field potential is a liner function, we are able to apply the principle of superposition. The electric field $\vec{E}$ is thus the the vector sum of the fields caused by the individual charges
$$ \vec{E} = \frac{1}{4\pi\epsilon_0} \sum_{k=1}^{n} \frac{q_k (\vec{R} - \vec{R'_k})}{ |\vec{R} - \vec{R'_k}|^3} \quad \left[\frac{V}{m}\right]$$

\subsection{Electric Field Due to a Continuous Distribution of Charge}
The electric field from a distribution of charge can be obtained by integrating the contribution of an element charge over the charge distribution. The differential element is the following:
$$ d\vec{E} = \vec{a_R}\frac{\rho dv'}{4\pi \epsilon_0 R^2} $$
Integrating this, we get:
$$ \vec{E} = \frac{1}{4\pi\epsilon_0} \int_{V'} \vec{a_R} \frac{\rho_v}{R^2} dv' =  \frac{1}{4\pi\epsilon_0} \int_{V'} \vec{a_R} \frac{dQ'}{R^2}  	\quad	\left[\frac{V}{m}\right] $$
This is integrating a charge over the volume. If we want to integrate over the surface, then:
$$ \vec{E} = \frac{1}{4\pi\epsilon_0} \int_{S'} \vec{a_R}\frac{\rho_s}{R^2} ds' = \frac{1}{4\pi\epsilon_0} \int_{S'} \vec{a_R}\frac{dQ'}{R^2} 	\quad	 \left[\frac{V}{m}\right] $$
Likewise, if we have a line charge:
$$ \vec{E} = \frac{1}{4\pi\epsilon_0} \int_{L'} \vec{a_R}\frac{\rho_l}{R^2} dl' =  \frac{1}{4\pi\epsilon_0} \int_{L'} \vec{a_R}\frac{dQ'}{R^2}\quad	 \left[\frac{V}{m}\right] $$

\subsection{Gauss's Law and Applications}
Gauss's Law follows from the divergence postulate of electrostatics by application of the divergence theorem. It states that the total outward flux of the electric field over any closed surface in free space is equal to the total charge enclosed in any surface divided by $\epsilon_0$.
$$ \oint_{S} \vec{ E}  \cdot ds = \frac{Q}{\epsilon_0} \quad \text{or} \quad \oint_{S} \vec{ D}  \cdot ds = Q $$
where $\vec{D}$ is related to $\vec{E}$ by $\vec{D} = \epsilon_r \epsilon_0 \vec{E}$

\subsection{Electric Potential}
We define a scalar potential $V$ such that 
$$ \vec{E} = -\nabla V$$
Electrical potential refers to the work done in carrying a charge from one point to another. To move a unit charge from a point $P_1$ to point $P_2$, work must be done against the electric field and is equal to 
$$ \frac{W}{q} = - \int \limits_{P_1}^{P_2} \vec{E} \cdot dl $$
This represents the difference in electric potential between the two points and thus, we have
$$ \Delta V = V_2 - V_1 = - \int \limits_{P_1}^{P_2} \vec{E} \cdot dl $$


\subsection{Dielectrics in Static Electric Field}

\subsection{Electric Flux Density and Dielectric Constant}

\subsection{Boundary Conditions for Static Electric Fields}

\subsection{Capacitors and Capacitance}

\subsection{Electrostatic Energy Forces}

\end{document}
