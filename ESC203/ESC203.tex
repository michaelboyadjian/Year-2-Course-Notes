\documentclass[11pt]{article}
\usepackage{fullpage}
\usepackage{amsmath}
\usepackage{graphicx}
\linespread{1.1}


\begin{document}

\title{ESC203 \\ Engineering and Society}
\author{Michael Boyadjian}
\maketitle
\pagebreak

\tableofcontents

\pagebreak


\section{Lectures}
\subsection{Lecture: 05 September 2019}

\subsubsection{Modernism}
Consisted of the following:
\begin{itemize}
\item Secularism
\item Observable Science
\item Colonialism
\item Consumerism
\item Capitalism
\item Industrialization
\end{itemize}
Some of the important figures of this era included:
\begin{itemize}
\item Descartes
\item Newton
\item Leibniz
\item Laplace
\end{itemize}

\subsubsection{Human Social Development}
From Ian Morris, \textit{Why the West Rules ... for now}
\begin{enumerate}
\item \underline{Energy Capture:} Per person calories obtained from the environment for food, home, etc.
\item \underline{Organization:} Size of  the largest city
\item \underline{War Making Capacity:} Number of troops, power, and speed of weapons
\item \underline{Information Technology:} Sophistication of available tools for sharing and processing data
\end{enumerate}

\subsubsection{Assumptions of Modernism}
\begin{enumerate}
\item Inevitable Progress
\item Valuing the Quantifiable and the Marketable
\item Perfectibility of the Human Project
\end{enumerate}

\subsubsection{Dark Side of Modernism}
\begin{itemize}
\item Devaluing the Non-Quantifiable
\begin{itemize}
\item \underline{Love} - the affective domain
\item \underline{Community} - social bonds and relationships
\end{itemize}
\item Denial of Negative Consequences
\begin{itemize}
\item \underline{Environment} - degradation, life cycle, etc.
\item \underline{Economics} - dismiss the free market losers
\end{itemize}
\end{itemize}

\subsection{Lecture: 11 September 2019}
\subsection{Lecture: 12 September 2019}
\subsection{Lecture: 18 September 2019}
\subsection{Lecture: 19 September 2019}
\subsection{Lecture: 19 September 2019}
\subsection{Lecture: 19 September 2019}
\subsection{Lecture: 02 October 2019}
\subsubsection{Values in Virtue Ethics}
\begin{itemize}
\item Emotions are integral and important part of moral perception
\item Motivation of the agent is crucial
\item No rigid rules; choices can be adapted to the situation and the people involved
\item Flexibility encourages the pursuit of creative solutions to tragic dilemmas
\item Tragic dilemmas can rarely be resolved to the complete satisfaction of all parties; any solution is likely to leave some remaiinder of pain

\end{itemize}
\pagebreak
\section{Readings}
\subsection{5 Things to Know About Technological Change}
\subsection{Do Artifacts Have Politics}
\subsection{Technological Momentum}
\begin{itemize}
\item Technological momentum offers an alternative to technological determinism and social construction
\item \textbf{Technological Determinism:} Belief that technical forces determine social and cultural changes
\item \textbf{Social Construction:} Social and cultural forces determine technical change
\item \textbf{Technological Momentum:} Social development shapes and is shaped by technology.
\item "Technology" refers to technological or sociotechnical systems.
\item Hughes not a technological determinist or social contructivist.
\item Technological momentum avoids the extremism of both other concepts
\item A technological system at times is a cause, at others an effect
\item Momentum is time dependant, and therefore not symmetrical over time
\item Some characteristics include acquired skill and knowledge, special-purpose machines and processes, enormous physical structures, and organizational bureaucracy
\item Provides the durability and the propensity for growth that were associated more commonly in the past with the spread of bureaucracy
\item Technological momentum can be located between the poles of social constructivism and technological determinism, providing a flexible mode of interpretation in accord with the history of large systems. 

\end{itemize}
\subsection{Scientific Revolutions}



\end{document}
