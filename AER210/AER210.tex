\documentclass[11pt]{article}
\usepackage{fullpage}
\usepackage{amsmath}
\usepackage{graphicx}
\linespread{1.1}


\begin{document}

\title{AER210 \\ Vector Calculus and Fluid Mechanics}
\author{Michael Boyadjian}
\maketitle
\pagebreak

\tableofcontents

\pagebreak
\part{Vector Calculus}
\section{Review of Multivariable Calculus}
\subsection{Vectors and Vector-Valued Functions}
\begin{itemize}
\item A vector is a quantity that has magnitude and direction
\item Vectors having the same length and direction are equivalent
\item \underline{Position Vector} - Initial point of the vector at the origin of the coordinate system
\end{itemize}

\subsection{Dot Product}
\begin{itemize}
\item $ \vec{a} \cdot \vec{b} \rightarrow$ scalar product
\end{itemize}
$$ \vec{a} \cdot \vec{b} = a_1b_1 + a_2b_2 + a_3b_3 \quad \quad (algebraic)$$
$$ \vec{a} \cdot \vec{b} = |a||b|cos\theta \quad \quad (geometric)$$

\subsection{Cross Product}
\begin{itemize}
\item $\vec{a} \times \vec{b} $ $\rightarrow$ another vector
\item $\vec{c}$ $=$ $\vec{a} \times \vec{b}$ is perpendicular to the plane containing $\vec{a}$ and $\vec{b}$
\end{itemize}
$$
\vec{a}\times \vec{b} = 
\begin{bmatrix}
i&j&k \\
a_1&a_2&a_3 \\
b_1&b_2&b_3 \\
\end{bmatrix}
\quad \quad
(algebraic)
$$
\smallskip
$$
|\vec{c}| = |\vec{a}\times \vec{b}| = |\vec{a}||\vec{b}|sin\theta \quad \quad (geometric)
$$
\begin{itemize}
\item $|\vec{c}|$ is the area of a parallelogram formed by $\vec{a}$ and $\vec{b}$
\end{itemize}
\subsection{Triple Scalar Product}
$$
\vec{a} \cdot (\vec{b} \times \vec{c}) = 
\begin{bmatrix}
a_1&a_2&a_3 \\
b_1&b_2&b_3 \\
c_1&c_2&c_3 \\
\end{bmatrix}
$$
\begin{itemize}
\item \underline{Volume of Parallelpiped} = $ |\vec{a}||\vec{b} \times \vec{c}|cos\theta = |\vec{a} \cdot (\vec{b} \times \vec{c})|$
\end{itemize}
\subsection{Vector Functions}
A vector function is a function hose input is a scalar (such as time) and output is a vector.
$$
r(t) = f(t)\vec{i} + g(t)\vec{j} + h(t)\vec{k}
$$
\subsubsection{Derivatives of Vector Functions}
\smallskip
$$
\frac{d\vec{r}(t)}{dt} = \vec{r}'(t) = \lim_{n\to0} \frac{\vec{r}(t+h)-\vec{r}(t)}{h} = f'(t)\vec{i} + g'(t)\vec{j} + h'(t)\vec{k}
$$
\\
At $h=0$, the vector $\frac{\vec{r}(t+h)-\vec{r}(t)}{h}$ approaches a vector that is tangent to the curve at point P.
\subsubsection{Unit Tangent Vector}
$$\vec{T}(\theta) = \frac{\vec{r}'(t)}{|\vec{r}'(t)|}$$
\subsubsection{Arc Length}
$$ S = \int_{a}^{b} |\vec{r}'(t)|dt = \int_{a}^{b} \sqrt{(x'(t))^2 + (y'(t))^2 + (z'(t))^2} dt
$$
\subsubsection{Partial Derivatives}
For functions of multiple variables, select the variable to differentiate with respect to, and treat all others as constants.
$$
f(x, b) = g(x) \quad \rightarrow \quad f_x(x, b) = g'(x)
$$
\medskip
\underline{example:} \\
Set $ f(x, y) = 3x^4y^2$. Find $f_x(x, y)$ and $f_y(x, y)$.
$$ f_x(x, y) = 12x^3y^2 \quad \quad f_y(x, y) = 6x^4y$$
\subsubsection{Higher Order Partial Derivatives}
$$(f_{x})_x = f_{xx} \quad \quad (f_{y})_y = f_{yy} \quad \quad (f_{x})_y = (f_{y})_x  = f_{xy} = f_{xy}$$
\subsubsection{Gradient Vector}
The gradient vector is the rate of change of a function $f(x,y,z)$ composed of its partial derivatives.
$$ \vec{\nabla}f = f_x\vec{i} + f_y\vec{j} + f_z\vec{k} $$
\subsubsection{Chain Rule}
$$ h = f(x,y,z) \quad \rightarrow \quad x=x(t) \quad y = y(t) \quad z = z(t)$$
$$ \frac{df(\vec{r}(t)}{dt} = \langle\frac{\partial f}{\partial x}, \frac{\partial f}{\partial y}, \frac{\partial f}{\partial z}\rangle \cdot \langle\frac{dx}{dt}, \frac{dy}{dt}, \frac{dz}{dt}\rangle
$$
$$ \frac{df(\vec{r}(t)}{dt} = \frac{\partial f}{\partial x}\frac{dx}{dt} + \frac{\partial f}{\partial y} \frac{dy}{dt} + \frac{\partial f}{\partial z}\frac{dz}{dt}
$$

\subsubsection{Tangent Planes}
$$ z - f(a,b) = f_x(a,b)(x-a) + f_y(a,b)(y-b)$$

\subsubsection{Linearization}
If $\Delta x  \rightarrow 0 \text{, } \Delta y \rightarrow 0 \text{, then } \Delta z \approx dz:$
$$ \Delta z = f_x(a,b)\Delta x + f_y(a,b)\Delta y$$

\pagebreak

\section{Multiple Integrals}

\subsection{Double Integrals Over Rectangles}

We can define a region, $R$, over which the integration will take place:
$$R = [a, b] \times [c,d] = \left \lbrace(x,y) \text{ } \epsilon \text{ } R^2 \text{ } |\text{ }  a\leq x \leq b \text{ , } c \leq y \leq d\right \rbrace $$
The \textbf{double integral} of $f$ over the rectangle $R$ is thus
$$\iint\limits_{R} f(x,y)dA = \lim_{\infty \text{, } a \to 0} \sum_{i=1}^{n} \sum_{j=1}^{n} f(x_{ij}, y_{ij}) \Delta A$$
If $f(x,y) \geq 0$, then the double integral could be interpreted as the volume in the defined region:
$$ V = \iint\limits_{R} f(x,y)dA$$
An iterated integral is defined as 
$$\int\limits_{a}^{b} \int\limits_{c}^{d} f(x,y)dydx = \int\limits_{a}^{b}\left[ \int\limits_{c}^{d} f(x,y)dy\right]dx$$
To evaluate the above, solve the inner integral with respect to $y$ treating $x$ as a constant. Then, solve the outer integral with respect to $x$ treating $y$ as a constant.
\\
\\
\textbf{Fubini's Theorem:} States if $f$ is continuous on region $R$, then 
$$\iint\limits_{R} f(x,y)dA = \int\limits_{a}^{b} \int\limits_{c}^{d} f(x,y)dydx = \int\limits_{c}^{d} \int\limits_{a}^{b} f(x,y)dxdy$$
If $f(x,y)$ can be expressed as $g(x)h(y)$, then we can express its integration as 
$$\iint\limits_{R} f(x,y)dA = \int\limits_{a}^{b} \int\limits_{c}^{d} g(x)h(y)dydx = \int\limits_{a}^{b} g(x)dx \int\limits_{c}^{d} h(y)dy$$


\subsection{Double Integrals Over General Regions}
Double integrals do not always cover rectangular regions. They often cover general regions bounded by several functions.
\\ \\
If $f$ is integrable over $R$, then we define the double integral of $f$ over $D$ by 
$$\iint\limits_D f(x,y)dA = \iint\limits_{R} F(x,y)dA =$$
In the case that $f(x,y) \geq 0$, we can still interpret $\iint\limits_D f(x,y)dA$ as the volume of the solid that lies above $D$ and under the surface $z = f(x,y)$.
\\ \\
Similarly to rectangular regions, we can evaluate iterated integrals with functions as parameters. \\ \\ If $f$ is a continuous function such that $D = \left\lbrace (x,y)\text{ } | \text{ } a \leq x \leq b \text{, } g_1(x) \leq y \leq g_2(x) \right\rbrace$ (\textbf{Type I}), then
$$\iint\limits_{D} f(x,y)dA = \int\limits_{b}^{a} \int\limits_{g_1(x)}^{g_2(x)} f(x,y)dy dx$$
If $f$ is a continuous function such that $D = \left\lbrace (x,y)\text { } | \text{ } c \leq y \leq d \text{, } h_1(y) \leq x \leq h_2(y) \right\rbrace$ (\textbf{Type II}), then
$$\iint\limits_{D} f(x,y)dA = \int\limits_{d}^{c} \int\limits_{h_1(y)}^{h_2(y)} f(x,y)dx dy$$
If regions do not overlap, they can also be summed:
$$\iint\limits_{D} f(x,y)dA  = \iint\limits_{D_1} f(x,y)dA + \iint\limits_{D_2} f(x,y)dA $$
Finally, if we integrate the constant function $f(x,y) = 1$, we'll simply get the area of $D$. 
$$\iint\limits_{D} 1dA  = A(D)$$

\subsection{Double Integrals in Polar Coordinates}
Recall that polar coordinates $(r, \theta)$ relate to rectangular coordinates $(x,y)$ with the following equations:
$$r^2 = x^2 + y^2 \quad \quad \quad x = r\cos\theta \quad \quad \quad y = r\sin\theta$$
We can define the region $R = \left\lbrace (r,\theta)\text{ }  | \text{ } a \leq r \leq b \text{, } \alpha \leq \theta \leq \beta \right\rbrace$. If $f$ is continuous on a polar rectangle defined by this region $R$, where $ 0 \leq \beta - \alpha \leq 2\pi $, then $$\iint\limits_{R} f(x,y)dA = \int\limits_{\alpha}^{\beta} \int\limits_{a}^{b} f(r\cos\theta,r\sin\theta)\cdot rdrd\theta$$
\subsection{Applications of Double Integrals}
\subsubsection{Density and Mass}
Consider a lamina with variable density $\rho(x,y) = \lim\frac{\Delta m}{\Delta A}$. We can obtain its total mass as the limiting value of its Riemann approximation:
$$ m = \lim_{k,l \to \infty} \sum_{i=1}^{l} \sum_{j=1}^{l} \rho (x_{ij}^{\star}, y_{ij}^{\star})\Delta A = \iint\limits_{D} \rho(x,y)dA  $$
\subsubsection{Moments and Center of Mass}
Consider a lamina that occupies a region $D$ and has density function $\rho (x,y)$. We can define the moments about the $x-axis$ and moments about the $y-axis$ as follows:
$$M_x = \iint\limits_{D} y\rho (x,y)dA \quad \quad \quad M_y = \iint\limits_{D} x\rho (x,y)dA$$
We can then say the coordinates of the centre of mass $(\bar x , \bar y)$ are  
$$ \bar x = \frac{M_y}{m} = \frac{1}{m}\iint\limits_{D} y\rho (x,y)dA \quad \quad \quad \bar y = \frac{M_x}{m} = \frac{1}{m}\iint\limits_{D} x\rho (x,y)dA$$
where $m$ is given by $m = \iint\limits_{D} \rho (x,y)dA$
\subsubsection{Moment of Inertia}
To find the moment of inertia about an area, the following integrals apply: 
$$I_x = \iint\limits_{D} y^2\rho (x,y)dA \quad \quad \quad I_y = \iint\limits_{D} x^2\rho (x,y)dA$$
The moment of inertia about the origin is given as 
$$ I_0 = \iint\limits_{D} (x^2 + y^2)\rho (x,y)dA$$
\subsection{Surface Area}
The area of a surface with equation $z = f(x,y)$, $(x,y) \in D$ where $f_x$ and $f_y$ are continuous , is 
$$ A(S) = \iint\limits_{D} \sqrt{[f_x(x,y)]^2 + [f_y(x,y)]^2+1} \text{ } dA $$
We can also use an alternative notation to express the same surface area function:
$$ A(S) = \iint\limits_{D} \sqrt{\left(\frac{\partial z}{\partial x}\right)^2 + \left(\frac{\partial z}{\partial y}\right)^2+1} \text{ } dA $$
\subsection{Triple Integrals}
The same way that single and double integrals were defined for one variable and two variable functions, we can define triple integrals for functions of three variables. \\ \\
We can consider $f$ which is defined as a rectangular box.
$$ B = \left\lbrace (x,y,z) \text{ } | \text{ } a \leq\ x \leq b \text{, } c \leq y \leq d \text{, } r \leq z \leq s \text{ }\right\rbrace$$ The triple integral then, of $f$ over box $B$ is 
$$\iiint\limits_{R} f(x,y,z)\text{ }dA = \lim_{l \text{, } m \text{, } n \to 0} \sum_{i=1}^{l} \sum_{j=1}^{m} \sum_{k=1}^{n} f(x_{ijk}^{\star}, y_{ijk}^{\star}, z_{ijk}^{\star}) \Delta V$$
\textbf{Fubini's Theorem (For Triple Integrals):} If $f$ is continuous over the rectangular box \\ $B = [a,b] \times [c,d] \times [r,s]$, then 
$$\iiint\limits_{R} f(x,y,z)\text{ }dV = \int\limits_{r}^{s} \int\limits_{c}^{d} \int\limits_{a}^{b} f(x,y,z)\text{ }dx\text{ }dy\text{ }dz $$
The same rules for double integrals can also be applied to triple integrals
\subsubsection{Triple Integrals in Cylindrical Coordinates}
We can convert to cylindrical coordinates $(r, \theta,z)$ from rectangular coordinates $(x,y,z)$ with the following equations:
$$r^2 = x^2 + y^2 \quad \quad \quad x = rcos\theta \quad \quad \quad y = rsin\theta \quad \quad \quad z=z$$
The formula for triple integrals in cylindrical coordinates then is:
$$\iiint\limits_{R} f(x,y,z)\text{ }dV = \int\limits_{\alpha}^{\beta} \int\limits_{h_1(\theta)}^{h_2(\theta)} \int\limits_{u_1(r\cos \theta, r\sin \theta)}^{u_2(r\cos \theta, r\sin \theta)} f(r\cos \theta,r \sin \theta,z)\cdot r \text{ }dz\text{ }dr\text{ }d\theta $$
To solve for center of mass questions, apply the following formulas: $$ m =\iiint\limits_{E} \rho(x,y,z)\text{ }dV \quad \quad \quad (\bar{x}, \bar{y}, \bar{z}) = \frac{1}{V}(M_{yz}, M_{xz}, M_{xy})$$
\subsubsection{Triple Integrals in Spherical Coordinates}
The spherical coordinates  $(\rho, \theta, \phi)$ of a point P in space is where $ \rho =|OP|$ is the distance from the origin to $P$, $\theta$ is the same angle as in cylindrical coordinates, and $\phi$ is the angle between the positive z-axis and the line segment $OP$.
$$ \rho \geq 0 \quad \quad \quad 0 \leq \phi \leq \pi$$ 
We can convert from rectangular to spherical coordinates with the following equations
$$\rho^2 = x^2 + y^2 + z^2 \quad \quad \quad x = \rho\sin\phi \cos \theta \quad \quad \quad y = \rho \sin \phi\sin\theta \quad \quad \quad z=\rho \cos \theta$$
The formula for triple integrals in cylindrical coordinates then is:
$$\iiint\limits_{R} f(x,y,z)\text{ }dV = \int\limits_{c}^{d} \int\limits_{\alpha}^{\beta} \int\limits_{a}^{b} f(\rho\sin\phi \cos \theta,\rho \sin\phi \sin \theta,\rho\cos\phi)\cdot \rho^2 \sin \phi \text{ }d\rho \text{ }d\theta \text{ }d\phi $$

\subsection{Change of Variables in Multiple Integrals}
Changing integration variables in multiple integrals is a bit of different then it is in single integrals. We consider the change of variables as a transformation from the (u,v)-plane to the (x,y)-plane:
$$ T(u,v) = (x,y)$$ where $x$ and $y$ are related by the equations
$$ x = g(u,v) \quad \quad \quad y=h(u,v)$$
We can define the Jacobian of the transformation given by the above equations as 
$$\begin{vmatrix}
\frac{\partial(x,y)}{\partial(u,v)}
\end{vmatrix} = \begin{vmatrix}
\frac{\partial x}{\partial v} & \frac{\partial x}{\partial u} \\
\frac{\partial y}{\partial u} & \frac{\partial y}{\partial v}
\end{vmatrix} = \frac{\partial x}{\partial u} \frac{\partial y}{\partial v} - \frac{\partial y}{\partial u} \frac{\partial x}{\partial v}
$$
With this definition, the formula for the change of variables in double integrals then is 
$$\iint\limits_{R} f(x,y)\text{ }dA = \iint\limits_{S} f(x(u,v),y(u,v))\begin{vmatrix}
\frac{\partial(x,y)}{\partial(u,v)}
\end{vmatrix}\text{ }du \text{ } dv$$
Similarly in triple integrals, the Jacobian and change of variables are the following
$$\begin{vmatrix}
\frac{\partial(x,y,z)}{\partial(u,v,z)}
\end{vmatrix} = \begin{vmatrix}
\frac{\partial x}{\partial u} & \frac{\partial x}{\partial v} & \frac{\partial x}{\partial w}  \\
\frac{\partial y}{\partial u} & \frac{\partial y}{\partial v}  & \frac{\partial y}{\partial w}  \\
\frac{\partial z}{\partial u} & \frac{\partial z}{\partial v} & \frac{\partial z}{\partial w} 
\end{vmatrix}
$$
$$\iint\limits_{R} f(x,y,z)\text{ }dA = \iint\limits_{S} f(x(u,v,w),y(u,v,w), z(x,y,z))\begin{vmatrix}
\frac{\partial(x,y,z)}{\partial(u,v,w)}
\end{vmatrix}\text{ }du \text{ } dv \text{ } dw$$

\pagebreak

\section{Vector Calculus}
\subsection{Line Integrals}
Line integrals are very similar to single integrals, except instead of integrating over an interval $[a,b]$, we integrate over a curve $C$. To evaluate a line integral, we can use the following formula: 
$$\int\limits_{C} f(x,y)\text{ }dA = \int\limits_{a}^{b} f(x(t),y(t)) \sqrt{\left(\frac{dx}{dt}\right)^2 + \left(\frac{dy}{dt}\right)^2}\text{ } dt$$
This is called the line integral with respect to arc length. We can also evaluate a line integral with respect to $x$ and $y$ using the following formulas:
$$\int\limits_{C} f(x,y)\text{ }dx = \int\limits_{a}^{b} f(x(t),y(t))\text{ } x'(t) \text{ } dt \quad \quad \quad \int\limits_{C} f(x,y)\text{ }dy = \int\limits_{a}^{b} f(x(t),y(t)) \text{ }y'(t) \text{ } dt$$
We can also parametrize a line segment which begins at $\vec{r_0}$ and ends at $\vec{r_1}$ using a vector representation:
$$ \vec{r}(t) = (1-t)\vec{r_0} + t\vec{r_1}$$
This helps to reduce the parameters of the integral to one variable $t$.
\\ \\
Line integral apply in vector fields in important equations such as that for work. If $F$ is a continuous vector field on a smooth curve C given by vector function $\vec{r}(t)$, $a\leq t \leq b$, then the line integral of $F$ along $C$ is 
$$\int\limits_{C} F \cdot dr = \int\limits_{a}^{b} F(\vec{r}(t))\cdot r'(t) \text{ } dt = \int\limits_{C} F \cdot T \text{ }ds$$
\subsection{Fundamental Theorem for Line Integrals}
\textbf{Theorem:} Let $C$ be a smooth curve given by the vector function $\vec{r}(t)$, $a \leq t \leq b$. Let $f$ be a differentiable function of two or three variable whose gradient vector $\nabla f$ is continuous on $C$. Then
$$\int\limits_{C} \nabla f \cdot dr = f(\vec{r}(b)) - f(\vec{r}(a))$$
We can say that $\int\limits_{C} \nabla f \cdot dr$ is independent of path in $D$ if and only if $\int\limits_{C} \nabla f \cdot dr = 0$ for every closed path $C$ in D. The line integrals of conservative vector fields are independent of path.
\\ \\ 
\textbf{Theorem:} Supposing that $F$ is a vector field that is continuous on an open connected region D. If $\int\limits_{C} \nabla f \cdot dr$ is independent of path on $D$, then $F$ is a conservative vector field on $D$, meaning there exists a function $f$ such that $ \nabla f = F$
\\ \\
\textbf{Theorem:} If $F(x,y) = P(x,y)\vec{i} + Q(x,y) \vec{j}$ is a conservative vector field on, where $P$ and $Q$ have continuous first order partial derivatives on a domain $D$, then  throughout $D$:
$$ \frac{\partial P}{\partial y} = \frac{\partial Q}{\partial x}$$ 
If the above is true, then $F$ is conservative

\subsection{Green's Theorem}
\textbf{Theorem:} Let $C$ is a positively oriented, piecewise smooth, simple closed curve  in the plane and let $D$ be the region bounded by $C$. If $P$ and $Q$ have continuous partial derivatives on an open region that contains $D$, then
$$ \int\limits_{C} P\text{ }dx +Q\text{ }dy = \iint\limits_{D} \left(\frac{\partial Q}{\partial x} - \frac{\partial P}{\partial y}\right) \text{ }dA$$
\subsection{Curl and Divergence}
\subsubsection{Curl}
If $F = P\vec{i} + Q\vec{j}+ R\vec{k}$ is a vector field on $R^3$ and the partial derivatives of $P$, $Q$, and $R$ all exist, then the curl of $F$ is the vector field defined by 
$$ curl\text{ } F = \left(\frac{\partial R}{\partial y} - \frac{\partial Q}{\partial z}\right) \vec{i} + \left(\frac{\partial P}{\partial z} - \frac{\partial R}{\partial x}\right) \vec{j} + \left(\frac{\partial Q}{\partial x} - \frac{\partial P}{\partial y}\right) \vec{k} = \nabla \times F$$
\textbf{Theorem:} If $F$ is a function that has continuous second-order partial derivatives, then
$$ curl(\nabla F) = 0$$
This also tells us that if $F$ is a conservative vector field, the $curl \text{ } F = 0$
\\ \\
\textbf{Theorem:} If $F$ is a vector field defined on all of $R^3$, whose component functions all have continuous partial derivatives and $curl \text{ } F = 0$, then $F$ is conservative.
\subsubsection{Divergence}
If $F = P\vec{i} + Q\vec{j}+ R\vec{k}$ is a vector field on $R^3$ and $\frac{\partial P}{\partial x}$, $\frac{\partial Q}{\partial y}$, and $\frac{\partial R}{\partial z}$ all exist then the divergence of $F$ is the function of three variables defined by:
$$ div \text{ } F = \frac{\partial P}{\partial x}\vec{i} +\frac{\partial Q}{\partial y}\vec{j} + \frac{\partial R}{\partial z}\vec{k} = \nabla \cdot F$$
If $F$ is a function that has continuous second-order partial derivatives, then
$$ div \text{ } curl \text { }F  = 0$$
\subsubsection{Vector Form of Green's Theorem}
The curl and divergence operators now allow us to write Green's theorem in vector form as follows:
$$ \oint\limits_{C} F \cdot dr = \iint\limits_{D} (curl \text{ } F) \cdot k \text { } dA \quad \quad \quad  \oint\limits_{C} F \cdot n \text { } ds = \iint\limits_{D} div \text{ } F(x,y) \text { } dA$$
\subsection{Parametric Surfaces and Their Areas}
Similar to how we can define parametric functions of one variable $t$, we can describe a surface $r(u,v)$ of two parameters, $u$, and $v$, where
$$ \vec{r}(u,v) = x(u,v)\vec{i}+y(u,v)\vec{j}+z(u,v)\vec{k}$$
If a smooth parametric surface $S$ is given by the equation $\vec{r}(u,v) = x(u,v)\vec{i}+y(u,v)\vec{j}+z(u,v)\vec{k}$ and $S$ is covered just once as $(u,v)$ ranges throughout the parameter domain $D$, then the surface area of $S$ is  $$A(S) = \iint\limits_{D} | \vec{r_u} \times \vec{r_v} | \text { } dA$$ where $\vec{r_u}$ and $\vec{r_v}$ are $$\vec{r_u} = \frac{\partial x}{\partial u}\vec{i} +\frac{\partial y}{\partial u}\vec{j} + \frac{\partial z}{\partial u}\vec{k} \quad \quad \quad \vec{r_v} = \frac{\partial x}{\partial v}\vec{i} +\frac{\partial y}{\partial v}\vec{j} + \frac{\partial z}{\partial v}\vec{k}$$
\subsection{Surface Integrals}
Using the parametrized function $\vec{r}(u,v)$ from before, if the components are all non-zero and non-parallel, then the surface integral is 
$$ \iint\limits_{S} f(x,y,z) \text{ } dS = \iint\limits_{D} f(\vec{r}(u,v)) \text { }| \vec{r_u} \times \vec{r_v} | \text { } dA$$
If $F$ is continuous vector field defined by a unit normal vector $\vec{n}$, then the surface integral of $F$ over $S$ is $$ \iint\limits_{S} F \cdot ds =  \iint\limits_{S} F \cdot \vec{n} \text { } ds$$ where $\vec{n}$ is $$\frac{\vec{r_u} \times \vec{r_v}}{|\vec{r_u} \times \vec{r_v}|} $$
This is also known as the \textbf{flux} of $F$ along $S$.
\subsection{Stoke's Theorem}
Stoke's Theorem can be considered a higher order version of Green's Theorem, where it relates a surface integral over a region $S$ to a line integral around the boundary of $S$.
\\ \\ 
\textbf{Theorem:} Let $S$ be an oriented piecewise-smooth surface that is bounded by a simple, closed, piecewise boundary curve $C$ with positive orientation. Let $F$ be a vector field whose components have continuous partial derivatives on an open region in $R^3$ that contains $S$. Then $$\int\limits_{C} F \cdot dr = \iint\limits_{S} curl \text{ } F \cdot dS = \iint\limits_{S} curl \text{ } F \cdot k \text{ } dA$$
\pagebreak

\part{Fluid Mechanics}
\section{Dimensional Analysis}
\subsection{Primary and Secondary Dimensions}
\subsection{Buckingham Pi Theorem}
\section{Introduction To Fluid Mechanics}
\subsection{Fluids}
\subsection{Statistical vs Continuum Approach}
\subsection{Forces}
\subsection{Viscosity}
\subsection{Compressibility}
\section{Hydrostatics}

\end{document}

