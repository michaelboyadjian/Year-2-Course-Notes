\documentclass[11pt]{article}
\usepackage{fullpage}
\usepackage{amsmath}
\usepackage{esint}
\usepackage{tabularx} 
\usepackage{cancel}
\usepackage{graphicx}
\linespread{1.1}


\begin{document}

\title{BME205 \\ Fundamentals of Biomedical Engineering}
\author{Michael Boyadjian}
\maketitle
\pagebreak

\tableofcontents

\pagebreak

\bigskip
\bigskip
\bigskip

\section{Foundation of Physiology}

\subsection{Homoeostasis}
Homoeostasis is the ability of a cell or organism to regulate its internal conditions typically using feedback systems to minimize variation and maintain health regardless of changes in external environment.

\subsubsection{Body Cells}
The fluid collectively contained within all body cells is known as \textbf{intercellular fluid (ICF)}. The fluid outside the cells is called \textbf{extracellular fluid (ICF)}. Extracellular fluid is made up of two components: \textbf{plasma}, the fluid component of blood; and \textbf{interstitial fluid}, which surrounds and bathes the cells

\subsubsection{Body Systems}
Homoeostasis is essential for the survival of each cell, and each cell, through its specialized activities, contributes as part of a body system to the maintenance of the internal environment shared by all cells. It is not a rigid, fixed state, or absolute setting, but rather a dynamic, steady state in which changes occur but are minimized by multiple dynamic equilibrium adjustment mechanisms. The following are factors that are homoeostatically regulated.
\begin{itemize}
\item Concentration of nutrient molecules 
\item Concentration of oxygen and carbon dioxide
\item Concentration of waste products
\item pH
\item Concentration of water, salt, and other electrolytes
\item Volume and pressure
\item Temperature
\end{itemize}
The 11 body systems also contribute to homoeostasis in the following ways:
\begin{itemize}
\item The \textbf{circulatory system} transports materials, such as nutrients, oxygen, carbon dioxide, wastes, electrolytes, and hormones from one part of the body to another.
\item The \textbf{digestive system} breaks down dietary food into small nutrient molecules that can be absorbed into the plasma for distribution to the body cells, and transfers water and electrolytes from the external environment to the internal environment.
\item The \textbf{respiratory system} receives oxygen from the external environment and eliminates carbon dioxide from the internal environment; it is important in regulating proper pH of the internal environment by adjusting the rate of acid-forming carbon dioxide. 
\item The \textbf{urinary system} removes excess water, salt, acid, and other electrolytes from the plasma and eliminates them in the urine.
\item The \textbf{skeletal system} provides support and protection for the soft tissues and organs. It also serves as a storage reservoir for calcium, an electrolyte whose plasma concentration must be maintained within very narrow limits.
\item The \textbf{muscular system} along with the skeletal system forms the basis of movement. The system enables an individual to move toward food or away from harm.
\item The \textbf{integumentary system} serves as an outer protecting barrier that prevents internal fluid from being lost from the body and foreign organisms from entering. It is also important in regulating body temperature.
\item The \textbf{immune system} defends against foreign invaders and body cells that have become cancerous and paves the way for replacing injured or worn-out cells
\item The \textbf{nervous system} is one of the two major regulatory systems of the body. This is especially important in detecting and initiating reactions to changes in the external environment. 
\item The \textbf{endocrine system} is the other major regulatory system. This is important in controlling the concentration of nutrients and, by adjusting kidney function, controlling the internal environment's volume and electrolyte composition.
\item The \textbf{reproductive system} is essential for perpetuating the species. Homoeostatic mechanisms ensure that both male and female reproductive systems are optimizes to favour reproductive success. 

\end{itemize}

\subsubsection{Homoeostatic Control Systems}
A homoeostatic control system is needed to maintain homoeostasis. This control system must be able to do three things:
\begin{enumerate}
\item Detect deviations from normal in the internal environment (receptor)
\item Integrate this information with any other relevant information (control centre)
\item Trigger the needed adjustments responsible for restoring this factor within the normal range (effector)
\end{enumerate}
These control systems can be grouped into two classes: \textbf{intrinsic} and \textbf{extrinsic} controls. Intrinsic controls are built into or are inherent in an organ. Most factors are however maintained by extrinsic controls, regulatory mechanisms initiated outside an organ to alter the activity of the organ.
\\ \\
To stabilize the necessary physiological factors, homoeostatic control systems must be able to detect and make necessary adjustments to various changes bringing feedback and feedforward loops into play. \textbf{Feedback} refers to responses made after a change has been detected. \textbf{Feedforward} describes responses made in anticipation of a change

\begin{itemize}
\item \textbf{Negative Feedback} \\ \\
Homoeostatic control systems operate primarily on the principle of negative feedback. A change in a homoeostatically controlled factor triggers a response seeking to maintain homoeostasis by moving the factor in the opposite direction of its original change, a corrective adjustment. This is structured as the following:

\item \textbf{Positive Feedback} \\ \\
The output enhances or amplifies a particular change so that the controlled factor continues to move in the direction of the initial change. This is much less frequent than negative feedback. Examples include child birth and heat stroke

\end{itemize}


\pagebreak

\section{Cell Physiology}
Most cells have three common subdivisions:
\begin{itemize}
\item \textbf{Plasma Membrane:}  Encloses the cells
\item \textbf{Nucleus:} Contains the cell's genetic material
\item \textbf{Cytoplasm:} Portion of the cell's interior not occupied by the nucleus but containing numerous organelles, structural proteins, transport and secretory vesicles
\end{itemize}
The following table summarizes the important structures of the cell.
\\ \\
\begin{tabularx}{\textwidth}{X|X|X}
\textbf{Cell Part} & \textbf{Structure} & \textbf{Function} \\
\hline
\textit{Organelles} \\
\hline
Endoplasmic Reticulum \\
Golgi Complex \\
Lysosomes \\
Centriole \\
Peroxisomes \\
Mitochondria \\
Vaults \\
\hline
\textit{Cytosol: Gel-like Portion} \\
\hline
Intermediary Metabolism Enzymes \\
Ribosomes \\
Transport, Secretort, and Endocytotic Vesicles \\
Inclusions \\
\hline
\textit{Cytosol: Cytoskeletal Portion} \\
\hline
Microtubules \\
Microfilaments \\
Intermediate Filaments \\
\end{tabularx}

\subsection{Cellular Metabolism}
\begin{itemize}
\item Intermediary metabolism refers collectively to the large set of of chemical reactions inside the cell that involve degradation, synthesis, and transformation of of small organic molecules, such as sugars, amino acids, and fatty acids. 
\item Critical for maintaining cell structure and cell growth. \item Occurs in the cytoplasm 
\end{itemize}

\subsection{Plasma Membrane}


\section{Central Nervous System}
\section{Peripheral Nervous System}

\end{document}
