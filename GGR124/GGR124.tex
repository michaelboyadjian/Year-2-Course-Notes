\documentclass[11pt]{article}
\usepackage{fullpage}
\usepackage{amsmath}
\usepackage{esint}
\usepackage{cancel}
\usepackage{graphicx}
\linespread{1.1}


\begin{document}

\title{GGR124 \\ Cities and Urban Life}
\author{Michael Boyadjian}
\maketitle
\pagebreak

\tableofcontents

\pagebreak

\bigskip
\bigskip
\bigskip

\section{Lecture 01: 07 January 2020}

\begin{itemize}
\item Introduction to the course
\item Assignments and deliverables were reviewed
\item No content was really covered.
\end{itemize}


\section{Lecture 02: 14 January 2020}

\subsection{Time Space Compression}
\begin{itemize}
\item The apparent compression of geographic space by faster means of transport and communication
\item Suggests that time is accelerated in such a way that space is shrunk
\item Highlights a relational rather than absolute conception of space
\item Relies on access to transportation and communications technologies, which is contingent on the existence of infrastructure and its maintenance, and economic, political and cultural restrictions on access
\item Does not occur naturally or automatically - it is the result of decisions and plans
\item Is political and affects people differently
\item Connects with the concepts of 
\textbf{power geometry}, which emphasizes the uneven experience of time-space compression globally
\end{itemize}

\subsection{Global Urbanization}
\begin{itemize}
\item Urban populations are growing twice as fast as the population in general
\item 55 \% of the world's population resides in urban areas
\item Continuing urbanization and  population growth is projected to add 2.5 billion people to the urban population by 2050, with nearly 90 per cent of the increase concentrated in Asia and Africa.
\end{itemize}

\subsection{Globalization}
\begin{itemize}
\item \textbf{Economic Globalization}
\begin{itemize}
\item Growth of transnational corporations and ‘offshoring’
\item Organization of production and markets at a global scale
\item Industrialization of the global south
\item Growing power of supranational financial institutions
\item Changing international division of labour
\item Rise of high-tech, biotech, new media, and other industries
\item Increased mobility of capital
\item Hanging communications and transportation technologies
\item Dramatic economic polarization (rise of the ’1\%')
\end{itemize}

\item \textbf{Political Globalization}
\begin{itemize}
\item The ‘hollowing out’ of nation-states 
\item Rise of supranational governing bodies 
\item Growing political power of cities
\item ‘Postnational citizenship’ and the formation of transnational elites \item Supranational trade agreements and ‘de-democratization’
\end{itemize}
\item \textbf{Cultural Globalization}
\begin{itemize}
\item Rise of social media and communications technologies
\item Movement of ideas, styles, music, film across borders
\item Decline of national identity and rise of regional, religious, linguistic, and other kinds of affinities
\end{itemize}
\end{itemize}

\subsection{Global Cities}
\begin{itemize}
\item The emerging global system relies on a network of urban place
\item This is changing the relationships between cities, as well as the role of cities in a globalizing economy
\item Cities have become the ‘command and control centres’of the global economy, and often compete with each other for investment
\item Globalization is also changing the internal social and organization of cities
\end{itemize}

\subsubsection{Economic Restructuring}
\begin{itemize}
\item Growth of primary cluster of high-level business service: finance, management, accounting, legal services, education, telecommunications, research
\item Growth of secondary cluster of employment which serves the first: real estate, construction, hotels, restaurants, luxury retail, private security, entertainment
\item Growth of third cluster in international tourism and hospitality
\item Decline of a fourth cluster of manufacturing employment. [Note: this is NOT necessarily the case in global south cities.]
\item Government services constitute a fifth cluster and serve to build, regulate, and reproduce the world city (planning, transportation, education, policing)
\item Growth of informal economies, precarious work, and chronic unemployment
\end{itemize}

\subsubsection{Social Restructuring}
\begin{itemize}
\item Growing social polarization following occupation restructuring
\item Political, economic, and cultural dominance of professionals and technocrats (1st and 2nd employment sectors)
\item Growth of a chronically underemployed ‘surplus population’ of growth of prisons that hold them (especially in US)
\item Growing social conflicts, often along the lines of race, in response to this polarization
\item Increasing reliance on police to manage social life 
\item Decline in public investment in social welfare in the name of competitiveness
\end{itemize}

\subsubsection{Physical Restructuring}
\begin{itemize}
\item Rapid growth of world city populations fueled by migration
\item Physical growth of cities to unprecedented size (L.A. as a functional unit with a radius of over 100km)
\item Gentrification of formerly industrial areas, downtowns, and waterfronts, and displacement of poorer populations
\item Growth of warehousing, big box, and other logistical spaces to support global supply chains
\item Spatial polarization deepens alongside social polarization with dramatic gaps in access to and quality of housing, parks, transit (ghettos and citadels)
\end{itemize}


\section{Lecture 03: 21 January 2020}

\subsection{Western Treaties}
\begin{itemize}
\item Ontario is covered by 46 treaties and agreements, signed between 1764 -1930. Some
groups never entered into treaties and negotiations are ongoing.
\item The terms of these treaties were understood differently by Indigenous people and settlers,
rooted in differing worldviews, with distinct concepts of land ownership.
\item First Nations peoples had (and still have) relations with the land that informed their
politics, spirituality and economy. Europeans saw the land as something to be owned and
exploited (private property).
\item Coupled with a language barrier and contrasting methods of knowledge transmission (oral
versus written), misunderstanding and outright deception characterized these relations.
\item Europeans began to impose artificial borders that do not line up with the traditional lands
or jurisdictions of Indigenous peoples, which span territory that spreads across provincial
lines and is located in both present-day Canada and the United States. 
\end{itemize}


\end{document}
