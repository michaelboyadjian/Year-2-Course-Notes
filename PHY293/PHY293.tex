\documentclass[11pt]{article}
\usepackage{fullpage}
\usepackage{amsmath}
\usepackage{graphicx}
\linespread{1.1}


\begin{document}

\title{PHY293 \\ Waves and Modern Physics}
\author{Michael Boyadjian}
\maketitle
\pagebreak

\tableofcontents

\pagebreak
\part{Waves}
\section{Simple Harmonic Oscillators}
\subsection{Equations of Motion}
If we look at a mass attached to a spring, we can model its motion using Hooke's Law:
$$ F = -kx_0$$
The negative indicates that this is a restoring force that brings it back to the origin. The constant $k$ shows that the force is proportional to the distance from the rest state.
\\ \\
Now if the spring is released, it must obey Newton's second law. Therefore the motion can be modelled as
$$ m \ddot x = -kx$$a second-order ordinary differential equation.
\subsection{General Solutions}
To consider the solution to this equation, let's define $\omega ^2 = \frac{k}{m}$, ($\omega = \frac{2\pi}{T}$ as well), which is called the \textit{angular frequency}. This allows us to represent this differential equation as $$ \ddot x + \omega ^2x = 0$$
Thus, the solution to this problem would be $$ x(t) = Acos(\omega t+\phi) \quad \quad \quad \text{or} \quad \quad \quad x(t) = A_1cos(\omega t) + A_2sin(\omega t)$$
where $A$ is the amplitude and $\phi$ is the phase shift, which can be determined by given initial values. Using what is given and our known equation, we can then solve for whatever unknown variables. It may also be important to remember that the velocity (first-derivative of position) is 
$$\dot x(t) = v(t) = -A\omega sin(\omega t + \phi )$$
\subsection{Energy}
Energy is an important tool to solve problems, simple harmonic oscillators being no exception. Without their dissipation, their mechanical energy stay the same forever. The total energy of a simple harmonic oscillator is simply the sum of its kinetic energy (KE) and potential energy (PE).
$$ E = KE + PE = \frac{1}{2}mv^2 + \frac{1}{2}kx^2 = \frac{1}{2}kA^2$$
which is a constant. The above equation shows us that the energy is proportional to the square of the amplitude.
\pagebreak
\section{Damped Harmonic Oscillators}
\subsection{Equations of Motion}
Realistically, we have to consider friction in the motion of a spring-mass system. This friction force is proportional to its velocity:
$$ F_d = -b \dot x$$
With this we can now define a new equation of motion:
$$ m\ddot x + kx + b\dot x = \ddot x + \gamma \dot x + \omega_0 ^2 x = 0$$ where $\gamma = \frac{b}{m}$ is called the \textit{damping factor} and $\omega_0 ^2$ is now the natural frequency of the oscillator - frequency if there was no damping. This tells us there could be multiple natural frequencies.
\subsection{General Solutions}
If we look at our equation from above, it's a second-order differential equation. The solution to it gives two roots, meaning that any solution can be a linear combination of both. This most general solution exists as
$$x(t) = a_pe^{r_pt}+ a_me^{r_mt}$$
where $r_p$ and $r_m$ are the roots of the equation:
$$r_p = -\frac{\gamma}{2}+\sqrt{\frac{\gamma^2}{4}-\omega_0^2}
\quad \quad \quad r_m = -\frac{\gamma}{2}-\sqrt{\frac{\gamma^2}{4}-\omega_0^2}$$
Note that this will not work if $\omega^2 = \frac{\gamma^2}{4}$, as we will see in conditions of damping.
\subsection{Conditions of Damping}
There are 3 cases of damping we can consider, which we can characterize as light damping ($\frac{\gamma^2}{4} < \omega_0^2$), critical damping ($\frac{\gamma^2}{4} = \omega_0^2$), and heavy damping ($\frac{\gamma^2}{4} > \omega_0^2$).
\subsubsection{Light Damping}
For light damping to occur, we consider the case where $\frac{\gamma^2}{4} < \omega_0^2$. When this happens, the general solution becomes $$x(t) = Ae^{\frac{\gamma t}{2}}cos( \omega t + \phi) \quad \quad  \rightarrow  \quad \quad \omega = \sqrt{\omega_0 ^2 - \frac{\gamma ^2}{4}}$$
A graph of the lightly damped model shows the decaying amplitude. We see that there is actually a factor by which this occurs 
$$ ln\left(\frac{A_n}{A_{n+1}}\right) = e^{\frac{\gamma T}{2}}$$ which is known as the \textbf{logarithmic decrement}
\subsubsection{Critical Damping}
This is an interesting case. Since $\omega_o^2 - \frac{\gamma^2}{4} = 0$, then $r_p = r_m = -\frac{\gamma}{2}$. Therefore, our general solution becomes
$$ x(t) = (a_p + a_m)e^{-\frac{\gamma t}{2}}$$ But this would not work as it could easily become an over constrained problem. So, the general solution actually becomes:
$$ x(t) = (A+Bt)e^{-\frac{\gamma t}{2}}$$
\subsubsection{Heavy Damping}
Heavy damping occurs when a system returns to its equilibrium without any oscillations. This is when $\frac{\gamma^2}{4} > \omega_0^2$. The general solution for this case is 
$$x(t) = e^{\frac{-\gamma t}{2}}(Ae^{\alpha t} + Be^{-\alpha t})$$
\subsection{Energy}
The energy of a damped oscillator eventually dissipates over time. We can find the rate at which this occurs by looking at the change in the total energy, which is given by: $$ E = KE + PE = \frac{1}{2}mv^2 + \frac{1}{2}kx^2$$ We find that the energy can be modelled by the equation:
$$E(t) = E_0e^{-\gamma t}$$
where $E_0$ is the initial energy at $t=0$ and $\frac{1}{\gamma}$ tells us the time it takes for the energy to decay by a factor of $e$. We can define $\tau = \frac{1}{\gamma}$ and show this equation as 
$$ E(t) = E_0e^{\frac{-t}{\tau}}$$ where $\tau$ is called the \textbf{decay time} or \textbf{time constant} of the system
\subsection{Quality Factor (Q)}
For energy saving purposes, we want a system to do as little driving as possible, meaning the less damping the better. To measure this, we define the Quality Factor (Q), measured by the ratio
$$Q = \frac{\text{tendency to oscillate}}{\text{tendency to damp}}$$
In our lightly damped system, oscillations last longer when $\gamma << 2\omega_0$, so a 'high quality' is achieved when $$ Q = \frac{\omega_0}{\gamma}$$
We can now write the pseudo angular frequency as 
$$ \omega_d^2 = \omega_0^2(1 - \frac{1}{4Q^2})$$
This can also be thought of as the number of oscillations an oscillator will achieve in a lifetime. If we define $\tau = \frac{1}{\gamma}$ and $n=\frac{\tau}{T_0}$, then the quality factor will be $$Q = 2\pi n$$
\pagebreak
\section{Forced Harmonic Oscillators and Resonance}
Remember the following for this chapter:
\begin{itemize}
\item If an oscillator is driven at frequency $\omega$, then it will oscillate at $\omega$, regardless of the values of $\omega_0$ and $\gamma$
\item If an oscillator is driven at $\omega \approx \omega_0$, then it will resonate
\item At resonance, the amplitude is approximately $A \approx QA_f$
\item The resonance frequency is exactly $\omega_0$ because it is where the peak in velocity response and power absorption are both located.
\end{itemize}

\subsection{Forced Undamped Oscillator}
Consider the SHO we discussed earlier, but this time with an engine attached to the system. If the oscillator is being driven at a frequency $\omega$ and applying a force with amplitude, $F_0$ , its equation of motion could be modelled as $$ \ddot x + \omega_0^2 x = \frac{F_0}{m} \cos (\omega t)$$
This is a non-homogeneous ODE. We don't consider the complimentary solution in this case, since this was covered in the unforced case. A particular solution to this problem is 
$$ x(t) = A(\omega)\cos (\omega t - \delta(\omega))$$
There are several cases to consider here:
\begin{itemize}
\item If $\delta = 0$, then $\cos \delta = 1$, so:
$$ A(\omega) = \frac{\omega_0^2}{\omega_0^2 - \omega^2} A_f \quad \quad \omega_0 > \omega $$
\item If $\delta = \pi$, then $\cos \delta = -1$, so:
$$ A(\omega) = -\frac{\omega_0^2}{\omega_0^2 - \omega^2} A_f\quad \quad \omega_0 < \omega $$
\end{itemize}
In the case where $\omega_0 = \omega$, the general solution from above cannot be applied. Instead, the solution is the following $$ x(t) = \frac{1}{2}A_f\omega_0 t \cos \left(\omega_0 t - \frac{\pi}{2}\right)$$
\subsection{Forced Damped Oscillator}
If we add damping to the forced oscillator system, we obtain the following equation of motion $$ \ddot x + \gamma \dot{x} + \omega_0^2 x = A_f \omega_0^2 \cos (\omega t)$$ 
We can consider the same general solution as the undamped case
$$ x(t) = A(\omega)\cos (\omega t - \delta(\omega))$$
where $A(\omega)$ and $\delta(\omega)$ are as follows
$$ A(\omega) = \frac{\omega_0^2}{\sqrt{\gamma^2 \omega^2 + (\omega_0^2-\omega^2)^2}} A_f \quad \quad \quad \delta(\omega) = \arctan \left(\frac{\gamma \omega}{\omega_0^2 - \omega^2}\right)$$

\subsection{Varying the Forcing Frequency}
It appears that resonance occurs when $\omega = \omega_0$, but this is in fact not the case. 
$$ \omega^2_{max} = \omega_0^2 - \frac{\gamma^2}{2} = \omega_0^2\left[1 - \frac{1}{2Q^2}\right]$$
The maximum amplitude is found when $\omega=\omega_{max}$
$$ A_{max} = A(\omega_{max}) = \frac{Q}{\sqrt{1 - \frac{1}{4Q^2}}} A_f $$
For practical purposes however, $\omega_{max} \approx \omega_0$ and $A_{max} \approx QA_f = \frac{F_0Q}{k}$

\subsection{Power Absorbed by Damping During Forced Oscillations}
The rate of energy loss due to damping is given as $$ - \dot{E} (\omega, t) = P (\omega, t) = bv^2 = bV^2(\omega)\sin^2(\omega t - \delta)$$ where $V(\omega)$ is the velocity response $$ V(\omega) = \frac{\omega_0\omega}{\sqrt{\gamma^2 \omega^2 + (\omega_0^2-\omega^2)^2}} A_f$$
The average power as a function of only $\omega$ is given as 
$$ \bar{P} (\omega) = \frac{F_0^2}{2m} \frac{\gamma}{\gamma^2+ (\frac{\omega_0^2}{\omega} - \omega)^2}$$
Friction absorbs the most energy when $\omega = \omega_0$. At this frequency, the average power becomes 
$$ \bar{P}_{max}  = \frac{F_0^2}{2\gamma m} =  \frac{F_0^2}{2b} $$ For this reason we call the natural frequency, $\omega_0$ the resonant frequency.
\pagebreak
\section{Coupled Harmonic Oscillators and Eigenmodes}
\subsection{Coupled Pendulums}
Consider two identical pendulums of length $L$ with a mass $m$ hanging from each. The masses are connected with a spring of stiffness $k$. There are two natural modes of motion to consider in this situation
\begin{itemize}
\item \textbf{Antisymmetric Normal Mode:}  Both pendulums oscillate in perfect sync , with the same amplitude and phase
$$ x_A(t) = x_B(t) = A \cos(\omega_1 t)$$
The spring is neither stretched or compressed and thus does not play any role in the motion. All elements of the system oscillate at the same frequency $\omega_1 = \omega_p$
\item \textbf{Symmetric Normal Mode:} The pendulums will oscillate 180 degrees out of phase and with a single angular frequency $\omega_2$
$$ x_A(t) = A \cos (\omega_2 t) \quad \quad \quad x_B(t) = A \cos (\omega_2 t + \pi)$$
This is considered a normal mode because both pendulums oscillate at the same frequency.
\end{itemize}
\subsection{General Solution Method}
Consider three masses coupled together with springs of spring constant $k$  with two fixed ends:
\begin{enumerate}
\item Define the equations of motion
$$ m\ddot{x}_A = -kx_A - k(x_A-x_B)$$
$$ m\ddot{x}_B = -k(x_B-x_A) -  k(x_B-x_C)$$
$$ m\ddot{x}_C = -k(x_C-x_B) - kx_C $$
\item Substitute $x(t) = A \cos (\omega t)$ and $ \ddot{x}(t) = -A\omega^2 \cos (\omega t)$ to simplify equations of motion 
$$ -A\omega^2m= -2kA + kB$$
$$ -B\omega^2m = -kB + kA +kC$$
$$ -C\omega^2m = -2kC +kB$$
\item Generate the matrix system of equations
$$ \begin{bmatrix}
2k - \omega^2m&-k&0 \\
-k&2k - \omega^2m&-k \\
0&-k&2k - \omega^2m 
\end{bmatrix}
\begin{bmatrix}
A\\
B\\
C
\end{bmatrix}
= \vec{0}$$
\item Take the determinant and find the eigenvalues
$$ \det{M} = (2k-\omega^2m)(2k-\omega^2m-k^2)(2k-\omega^2m+k^2)$$
$$ \omega_1 = \sqrt{\frac{2k}{m}} \quad \quad \quad \omega_2 = \pm \sqrt{\frac{(2-\sqrt{2})k}{m}} $$
\item Compose general solution and solve for the constants using the initial conditions
$$x(t) = C\cos(\omega_1 t) + D\cos(\omega_2 t) $$
\end{enumerate}
\pagebreak
\section{Wave Equation and Standing Waves}
\subsection{Wave Equation}
The wave equation is given as the following 
$$ \frac{\partial ^2 y}{\partial t^2} = \nu^2\frac{\partial^2 y}{\partial x^2}$$
\subsection{Standing Waves}
The general equation for standing waves can be given as such that the variables are separable
$$ y(x,t) = f(x)h(t)$$ 
Therefore, the general solution for a standing waves on a string is
$$ y_n(x,t) = A_n \sin \left(\frac{n\pi x}{L}\right) \cos (\omega _n t)$$
where $n$ is known as the harmonic number. 
\\ \\
Some important things to note about standing waves:
\begin{itemize}
\item A \textbf{node} is a position where there is zero displacement (the string doesn't move)
\item An \textbf{antinode} is the position where the largest displacement or amplitude occurs
\item The mode with no nodes is known as the \textbf{fundamental} mode or \textbf{first harmonic}
\item The $n^{th}$ harmonic has $n-1$ nodes and $n$ antinodes
\item The wavelength of the $n^{th}$ mode is $\lambda_n = \frac{2L}{n}$ and each has a corresponding wave number, $k_n = \frac{2\pi}{\lambda_n} = \frac{n\pi}{L}$
\end{itemize}
\subsection{Energy}
The energy per unit length of a string is given as
$$E = \frac{1}{2} \mu \int_0^L \left[ \left(\frac{\partial y}{\partial t}\right)^2 + \nu^2 \left(\frac{\partial y}{\partial t}\right)^2 \right] dx $$
For a single mode, the energy equation is simplified to 
$$ E_n = \frac{1}{4}\mu L \omega_n^2A_n^2$$ And the total energy is just the sum of the energies of each mode
$$ E = \sum_{n=1}^{\infty} E_n = \frac{1}{4}\mu L \sum_{n=1}^{\infty} \omega_n^2A_n^2$$
\pagebreak
\section{Travelling Waves}
The main travelling wave function that will be encountered is the following 
$$y(x,t) = A \sin (kx \pm \omega t)$$ 
The relationship  $v = \frac{\omega}{k} = \lambda \nu$ may also be useful
\subsection{Solutions to the Wave Equation}
The most general solution to the wave equation is $$y(x,t) = f(x-vt) + g(x-vt)$$
\subsection{Reflection and Transmission}
When a wave hits discontinuity part of it is transmitted and part of it is reflected. This discontinuity occurs in the phase speed. If the mass density, $\mu$ or the tension, $t$ changes suddenly, the phase speed will as well. Therefore, there are two phase speeds
$$ v_1 =\sqrt{ \frac{T}{\mu_1}} \quad x<0 \quad \quad \quad v_2 =\sqrt{ \frac{T}{\mu_2}} \quad x>0 $$
We find that the amplitude of the transmitted wave is equal to the sum of the amplitudes of the incident and reflected waves in the following relation
$$ A_I + A_R = A_T$$
From this we can derive the transmission and reflection coefficients, $\tau_{12}$ and $\rho_{12}$
$$ \tau_{12} = \frac{2k_1}{k_1+k_2} = \frac{2\sqrt{\mu_1}}{\sqrt{\mu_1}+\sqrt{\mu_2}}$$
$$ \rho_{12} = \frac{k_1-k2}{k_1+k_2} = \frac{\sqrt{\mu_1}-\sqrt{\mu_2}}{\sqrt{\mu_1}+\sqrt{\mu_2}}$$

\pagebreak
\section{Wave Packets and Dispersion}

If the phase speed depends on the wavenumber or frequency of the wave, this is said to be in a dispersive medium. In this case,
$$ \frac{\omega}{k} = v(\omega)$$
The important part in this chapter is being able to find the group velocity, which is the following
$$ v_g = \frac{d\omega}{dk}$$ 
\pagebreak

\end{document}


\part{Modern Physics}
\pagebreak
\section{Special Relativity}
\section{Photoelectric Effect}
\section{Compton Scattering}
\section{Line Spectra}
\section{Double-Slit Diffraction}
\section{Wave-Particle Duality}
\section{Uncertainty Principle}
\section{Wavefunctions}
\section{Schrödinger Wave Equation}
The Schrödinger wave equation in its time-dependent form for a particle moving in a potential $V$ in one dimension is given by
$$i\bar{h} \frac{\partial \Psi (x,t) }{\partial t} = - \frac{\bar{h}^2}{2m} \frac{\partial ^2\Psi (x,t) }{\partial t^2} + V\Psi (x,t) $$

\subsection{Normalization and Probability}
The probability of the particle being between $x_1$ and 
$x_2$ is given by
$$P = \int_{x_1}^{x_2} \Psi * \Psi dx $$
The wave function must also be normalized so that the probability of the particle being somewhere on the axis is 1
$$ \int_{-\infty}^{\infty} \Psi (x,t) * \Psi (x,t) dx  =1$$





