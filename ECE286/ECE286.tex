\documentclass[11pt]{article}
\usepackage{fullpage}
\usepackage{amsmath}
\usepackage{amssymb}
\usepackage{esint}
\usepackage{cancel}
\usepackage{graphicx}
\linespread{1.1}


\begin{document}

\title{ECE286 \\ Probability and Statistics}
\author{Michael Boyadjian}
\maketitle
\pagebreak

\tableofcontents

\pagebreak

\bigskip
\bigskip
\bigskip


\section{Introduction to Statistics and Data Analysis}

\subsection{Definitions}
\begin{itemize}
\item \textbf{Probability:} Mathematical theory that models and analyses uncertainty based on 3 fundamental axioms
\item \textbf{Statistics:} Empirical; based on data and measurements; we infer characteristics of a phenomenon
\end{itemize}

\subsection{Set Theory}
A set is a set of objects (or elements) represented using capital letters $A$, $B$, $C$, $S$
\begin{itemize}
\item \textbf{Universal Set:} Set of all elements or objects of interest
\item \textbf{Empty Set:} Set without any elements, denoted $\varnothing$
\item \textbf{Compliment of a Set:} $A' = \{x | x \notin A\} $
\item \textbf{Union of Two Sets:} $ A \cup B = \{x | x \in A \text{ or } x \in B\}$
\item \textbf{Intersection of Two Sets:} $ A \cap B = \{x | x \in A \text{ and } x \in B\}$
\item \textbf{Disjoint Sets:}  Also called mutually exclusive; $A$ and $B$ are disjoint iff  $A \cap B = \varnothing$
\item \textbf{Subsets:} Denoted $A \subset B$; $A$ is a subset of $B$ iff $x \in A \implies x \in B$
\end{itemize}

\subsection{Counting}
There are three different methods of counting
\begin{enumerate}
\item \textbf{With Replacement, With Ordering}
\begin{itemize}
\item Given $n$ distinct objects pick one, note down its kind, and put it back
\item Repeat $k$ times
\item Produces $n^k$ possibilities
\item \textit{ex.} Number of 8-character passwords $= 62^8$
\end{itemize}
\item \textbf{Without Replacement, With Ordering}
\begin{itemize}
\item Same as previous, but no replacement
\item Produces $\frac{n!}{(n-k)!}$ possibilities
\item Describes number of \textbf{permutations}, denoted $_nP_k$
\item \textit{ex.} Number of 8-character passwords without repeating characters $= \frac{62!}{54!}$
\end{itemize}
\bigskip
\item \textbf{Without Replacement, Without Ordering}
\begin{itemize}
\item Ordering does not matter, two sequences with same objects count as the same
\item Produces $\frac{n!}{k!(n-k)!}$ possibilities
\item Describes number of \textbf{combinations}, denoted $_nC_k = {n \choose k}$
\item \textit{ex.} Number of ways to pick 5 good chips and 3 bad chips out of 90 good and 10 defective $= {90 \choose 5}{ 10 \choose 3}$
\end{itemize}
\end{enumerate}

\pagebreak

\section{Probability}
Probability theory rests on the notion of a random experiment. We don't know what the outcome will be until we actually run the experiment. There are three kinds of experiments:
\begin{itemize}
\item Designed
\item Observational
\item Retrospective
\end{itemize}
An experiment has a procedure and a measurement and always produces one outcome

\subsection{Sample Spaces}
The sample space describes the set of all possible outcomes  $(S)$. There are three different kinds of sample spaces
\begin{itemize}
\item Finite $\rightarrow$ discrete
\item Countably Infinite $\rightarrow$ discrete
\item Continuous
\end{itemize}

\subsection{Events}
An event is the set of outcomes we are interested in. This must be a subset of the sample space. The \textbf{event class} is the set of all subsets of the sample space. Probabilities will be assigned to the events in $\varepsilon$

\subsection{Relative Frequency}
We want to assign probabilities consistent with the idea of relative frequency, which is defined as 
$$\lim_{n \to \infty} \frac{n_A}{n} = p_A$$
This is only useful to model the likelihood of events occurring, so probabilities are assigned using probability axioms

\subsection{Axioms of Probability}
\begin{enumerate}
\item For any event $A$, $P(A) \geq 0$
\item $P(S) = 1$
\item For any two events $A$ and $B$ such that $P(A \cap B) = \varnothing$, then $ P(A\cup B) = P(A) + P(B)$
\end{enumerate}

\subsection{Properties}
\begin{enumerate}
\item $P(A') = 1 - P(A)$
\item $P(A) \leq 1$
\item $P(\varnothing) = 0$
\item For any events $A$ and $B$: $P(A \cup B) = P(A) + P (B) - P(A \cap B)$
\end{enumerate}

\subsection{Conditional Probability}
Conditional probability describes the probability of some event $B$ occurring given an event $A$ has occurred. This is denoted as $P(B|A)$, which is read as \textit{"the probability of $B$ given $A$"}. \\ \\ A conditional probability relative to a subspace $A$ of $S$ may also be calculated directly from the probabilities assigned to the elements of the original sample space $S$
$$ P(B|A) = \frac{P(A \cap B)}{P(A)}$$

\subsubsection{Independent Events}
Two events are said to be independent if and only if 
$$P(B|A) = P(B) \quad \quad \text{or} \quad  \quad P(A|B) = P(A)$$
$A$ and $B$ are otherwise dependent

\subsubsection{Product Rule and Multiplicative Rule}
If in an experiment, both events $A$ and $B$ can occur, then 
$$ P(A \cap B) = P(A)P(B|A)$$
We can use this to find another expression for the independence of events. Two events $A$ and $B$ are independent if and only if 
$$ P(A \cap B) = P(A)P(B)$$
This multiplicative rule can be extended to more than just two-event situations. If, in an experiment, the events $A_1$, $A_2$, ... $A_k$ occur, then
$$ P(A_1 \cap A_2 \cap \cdots \cap A_k) = P(A_1)P(A_2|A_1)P(A_3|A_1 \cap A_2) \cdots P(A_k| A_1 \cap A_2 \cap \cdots A_{k-1})$$ If these events are independent, then 
$$ P(A_1 \cap A_2 \cap \cdots \cap A_k) = P(A_1)P(A_2)P(A_3) \cdots P(A_k)$$

\subsection{Bayes' Rule}

\pagebreak

\section{Random Variables, Probability Distributions and Expectation}
\section{Discrete and Continuous Probability Distributions}
\section{Functions of Random Variables}
\section{Estimation Problems}
\section{Hypothesis Testing}
\section{Simple Linear Regression and Correlation}


\end{document}
